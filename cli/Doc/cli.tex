\documentclass[11pt]{article} 

%------------------------------
%structure input 
%------------------------------
%------------------------------
%commentpackage 
%------------------------------
\usepackage{lastpage}
\usepackage{graphicx}
\setlength\parindent{0pt}
\usepackage[most]{tcolorbox}
\usepackage{booktabs}
\usepackage{etoolbox}

%------------------------------
%font settings 
%------------------------------
\usepackage[utf8]{inputenc} %required for inputting international characters,
\usepackage[T1]{fontenc} % output font encoding for international characters,
\usepackage[sfdefault,light]{roboto} %use roboto font

%------------------------------
%page setting 
%------------------------------
\usepackage{geometry}

\geometry{
  paper=a4paper,                % Change to letterpaper for US letter,        
  top=3cm,                      % Top margin,
  bottom=3cm,                   % Bottom margin,
  left=2.5cm,                   % Left margin, 
  right=2.5cm,                  % Right margin,
  headheight=14pt,              % Header height,
  footskip=1.4cm,               % Space from the bottom margin to baseline of the footer,
  headsep=1.2cm,                % Space from the top margin to the baseline of the header,
  %showframe,                   % Ucomment to show how the type block is set on the page,
}

%------------------------------
%header and footer
%------------------------------
%packages and parameters;
 \usepackage{fancyhdr}
 \pagestyle{fancy}

 % left header; 
 % output the instructor in brackets if one was set
 \lhead{\small\assignmentClass\ifdef{\assignmentClassInstructor}{\ (\assignmentClassInstructor):}{}\ \assignmentTitle} 
 \chead{}

 % right header; 
 % output the author name if one was set, otherwise the due date if that was set
 \rhead{\small\ifdef{\assignmentAuthorName}{\assignmentAuthorName}{\ifdef{\assignmentDueDate}{Due\ \assignmentDueDate}{}}} 

 % footer settings
 \lfoot{} % Left footer
 \cfoot{\small Page\ \thepage\ of\ \pageref{LastPage}} % Centre footer
 \rfoot{} % Right footer
 \renewcommand\headrulewidth{0.5pt} % Thickness of the header rule

%------------------------------
%modify section styles 
%------------------------------
\usepackage{titlesec} % required for modifying sections

%-----------------
%section settings
\titleformat
{\section}
[block]
{\Huge\bfseries}
{\assignmentQuestionName \thesection:}
{6pt}
{}

\titlespacing{\section}{0pt}{1\baselineskip}{2\baselineskip} % spacing around section titles, the order is: left, before and after

%-----------------
% subsection

\titleformat
{\subsection} % section type being modified
[block] % shape type, can be: hang, block, display, runin, leftmargin, rightmargin, drop, wrap, frame
{\huge\bfseries} % format of the whole section
{\thesubsection} % format of the section label
{4pt} % space between the title and label
{} % code before the label

\titlespacing{\subsection}{0pt}{0.5\baselineskip}{0.5\baselineskip} % spacing around section titles, the order is: left, before and after

%-----------------
% subsubsection

\titleformat
{\subsubsection} % section type being modified
[block] % shape type, can be: hang, block, display, runin, leftmargin, rightmargin, drop, wrap, frame
{\LARGE\bfseries} % format of the whole section
{\thesubsubsection} % format of the section label
{3pt} % space between the title and label
{} % code before the label

\titlespacing{\subsubsection}{0pt}{0.5\baselineskip}{0.5\baselineskip} % spacing around section titles, the order is: left, before and after



%------------------------------
%Title  
%------------------------------
\author{\textbf{\textit{\huge \assignmentAuthorName}}}
\date{}
\title{
  \thispagestyle{empty}
  \vspace{0.1\textheight}
  \textbf{\Huge\assignmentClass:\ \assignmentTitle}\\[-3pt]
  \ifdef{\assignmentDueDate}{{\large Due\ on\ \assignmentDueDate}\\}{}
  \ifdef{\assignmentClassInstructor}{{\large \textit{\assignmentClassInstructor}}}{}
  \vspace{0.32\textheight}
}

%------------------------------
%lstlisting setting 
%------------------------------
%\usepackage[T1]{fontenc}
%\usepackage{textcomp}
\usepackage{listings}
% 在LaTex中添加代码
\usepackage{color}
 
\definecolor{codegreen}{rgb}{0,0.6,0}
\definecolor{codegray}{rgb}{0.8,0.8,0.8}
\definecolor{codepurple}{rgb}{0.58,0,0.82}
\definecolor{backcolour}{rgb}{0.95,0.95,0.92}

\lstset{
  backgroundcolor=\color{backcolour},
  commentstyle=\color{codegreen},
  %keywordstyle=\color{magenta},
  stringstyle=\color{codepurple},
  basicstyle=\sffamily\footnotesize,
  breakatwhitespace=false,
  breaklines=true,
  captionpos=b,
  keepspaces=true,
  frame=single,
  numbers=left,
  numbersep=5pt,
  numberstyle=\tiny,
  showspaces=false,
  showstringspaces=false,
  showtabs=false,
  tabsize=2
}

%------------------------------
%newcommand and environment
%------------------------------
\newcommand{\answer}[1]{
  \begin{tcolorbox}[breakable,enhanced,colback=codegray]
    #1
  \end{tcolorbox}
}

\newenvironment{question}{
   \vspace{0.5\baselineskip} %whitespace before the question
   \section{} %blank section title (e.g. just Question 2)
   % set the left footer to state the question continues on the next page, this is reset to nothing 
   \lfoot{\small\itshape\assignmentQuestionName~\thesection~continued on next page\ldots} 
 }{
   \lfoot{} % reset the left footer to nothing if the current question does not continue on the next page
}

%------------------------------
%Tikz & PGF Setting
%------------------------------
\usepackage{tikz,pgfplots}
\usepackage{subcaption}

\pgfplotsset{
  standard/.style={
   axis x line=middle,
   axis y line=middle,
   enlarge x limits=0.15,
   enlarge y limits=0.15,
   %every axis x label/.style={at={(current axis.right of origin)},anchor=north west},
   %every axis y label/.style={at={(current axis.above origin)},anchor=east},
   every axis plot post/.style={mark options={fill=black}}
  }
}


%------------------------------
%Equation Order Setting to prefix the equation order.
%------------------------------
\makeatletter
\@addtoreset{equation}{section}
\makeatother
\renewcommand{\theequation}{\arabic{section}.\arabic{equation}} 

%------------------------------
%Chinese Environment 
%------------------------------
\usepackage{CJK}
\newcommand{\chinesebox}[1]{
  \begin{tcolorbox}[breakable,enhanced,colback=codegray]
   \begin{CJK*}{UTF8}{gbsn}
      #1
   \end{CJK*} 
  \end{tcolorbox}
}

\newcommand{\chinese}[1]{
 \begin{CJK*}{UTF8}{gbsn}
   #1
 \end{CJK*}
}


%------------------------------
%Multirows to Done things: 
%------------------------------
\usepackage{multirow}
\renewcommand{\multirowsetup}{\centering}


%------------------------------
%Ams math definition
%------------------------------
\usepackage{amsmath}


%------------------------------
%Frame definitions
%------------------------------
\usepackage{tcolorbox} 


%------------------------------
%Table setting
%------------------------------
\usepackage{array}
\renewcommand\arraystretch{1}





%------------------------------
%Input Parameters 
%------------------------------
% Required
\newcommand{\assignmentQuestionName}{Part }
\newcommand{\assignmentClass}{Command Line} % Course/class
\newcommand{\assignmentTitle}{CLI\ \#1} % Assignment title or name
\newcommand{\assignmentAuthorName}{XieYuhan} % Student name

% Optional (comment lines to remove)
\newcommand{\assignmentClassInstructor}{XieYuhan 17:44 } % Intructor name/time/description
\newcommand{\assignmentDueDate}{2023-12-30 } % Due date

\begin{document}

\maketitle

\Large

%------------------------------
%Preview sec1
%------------------------------
\clearpage
\section{Preview}%
\label{sec:preview}

This document aims to achieve the basic implementation of commmand line system utilized by mlpack, file related:

\begin{center}
  \begin{tabular}{|c|c|}
   \hline
    \textbf{File}& \textbf{Func}  \\
   \hline
    cli.cpp &  \\
   \hline
    cli.hpp &  \\
   \hline
    cli\_impl.hpp & \\
   \hline
    option.cpp &  \\
   \hline
    option.hpp &  \\
   \hline
    option\_impl.hpp & \\
   \hline
  \end{tabular}
\end{center}

This cli.hpp 

%------------------------------
%end of Preview sec1
%------------------------------

%------------------------------
%Program Doc sec
%------------------------------
\clearpage
\section{Program Doc}%
\label{sec:program_doc}

%------------------------------
%Procession sub2.1
\subsection{Procession}%
\label{sub:procession}

Use following macro  to declare the program docutmentation instance:
\begin{lstlisting}[language=c++,label=lst:0lst,caption=Data Structure(\textbf{cli.hpp})]
#define PROGRAM_INFO(NAME, DESC) static mlpack::util::ProgramDoc \
    io_programdoc_dummy_object = mlpack::util::ProgramDoc(NAME, DESC);
\end{lstlisting}

Then prototype a \textbf{ProgramDoc} class, and when constructor get called, \textbf{CLI} class get constructed, which will generate a singleton pointer:

\begin{lstlisting}[language=c++,label=lst:1lst,caption=ProgramDoc(\textbf{option.hpp})]
/**
 * @Class ProgramDoc
 * 
 * @Construct To construct CLI and assign the value 
 *            of two its member;
 * @Mem1 programName     Storing program name;
 * @Mem2 documentation   Storing program documentation;
 */
class ProgramDoc
{
 public:
  ProgramDoc(const std::string& programName,
             const std::string& documentation);

  std::string programName;
  std::string documentation;
};
\end{lstlisting}

When following constructor is called, tie this instance of class \textbf{ProgramDoc} to the class singleton \textbf{CLI}:

\begin{lstlisting}[language=c++,label=lst:2lst,caption=ProgramDoc Constructor(\textbf{option.cpp})]
ProgramDoc::ProgramDoc(const std::string& programName,
                       const std::string& documentation) :
    programName(programName),
    documentation(documentation)
{
  // Register this with CLI.
  CLI::RegisterProgramDoc(this);
}
\end{lstlisting}
Call \textbf{GetSingleton} and tie \textbf{ProgramDoc} to it:
\begin{lstlisting}[language=c++,label=lst:3lst,caption=ProgramDoc Constructor 2(\textbf{option.cpp})]
void CLI::RegisterProgramDoc(ProgramDoc* doc){
  if (doc != &emptyProgramDoc)
    GetSingleton().doc = doc;
}
\end{lstlisting}

%-----end of Procession sub2.1

%------------------------------
%Data Structure sub2.2
\subsection{Data Structure}%
\label{sub:data_structure}

\begin{center}
  \begin{tabular}{|p{3cm}|p{6cm}|p{6cm}|}
    \hline
     \textbf{ProgramDoc} & \textbf{Member}& \textbf{Desc} \\
    \hline
      constructor& ProgramDoc & Constructor; \\
    \hline
      & programName & Program name; \\
    \hline
      & documentation & Program documentation;\\
    \hline
  \end{tabular}
\end{center}


\begin{center}
  \begin{tabular}{|p{3cm}|p{6cm}|p{6cm}|}
    \hline
     \textbf{ProgramDoc} & \textbf{Member}& \textbf{Desc} \\
    \hline
     & CLI() & Constructor; \\
    \hline
     &  util::ProgramDoc *doc& Pointer to ProgramDoc\\
    \hline
     & RegisterProgramDoc & Register tie ProgramDoc to the CLI class;\\
    \hline
     & GetSingleton & Get the singleton pointer of the CLI class; \\
    \hline
  \end{tabular}
\end{center}


\begin{lstlisting}[language=c++,label=lst:1lst,caption=CLI class]
class ProgramDoc{
 public:
  ProgramDoc(const std::string& programName,
             const std::string& documentation);

  std::string programName;
  std::string documentation;
}

class CLI{
  ...

public:
  util::ProgramDoc *doc; // 
  static void RegisterProgramDoc(util::ProgramDoc* doc);
  static CLI& GetSingleton();
  static CLI* singleton;

private: 
  CLI();
  CLI(const std::string& optionsName);
  CLI(const CLI& other);

  ...
};
\end{lstlisting}

%-----end of Data Structure sub2.2

%------------------------------
%end of Program Doc sec
%------------------------------



%------------------------------
%Skills sec -1
%------------------------------
\clearpage
\section{Skills}%
\label{sec:skills}

%------------------------------
%Singleton sub-1.1
\indent
\subsection{Singleton}%
\label{sub:singleton}

Using following code to enable a singleton:

\begin{lstlisting}[language=c++,label=lst:100lst,caption=Singleton]
class Singleton{
  public:
    Singleton* singleton;
    Singleton& GetSingleton():
}
\end{lstlisting}

Utilization of the singleton, if you use a singleton then the pointer should be initialized globally;
\begin{lstlisting}[language=c++,label=lst:101lst,caption=Singleton]
// Variable singleton prototype;
Singleton* Singleton::singleton = NULL;

// Implemented Singleton::GetSingleton();
//     returning the sole instance of this class.
CLI& CLI::GetSingleton()
{
  if (singleton == NULL)
    singleton = new CLI();

  return *singleton;
}
\end{lstlisting}




%-----end of Singleton sub-1.1


%------------------------------
%end of Skills sec -1
%------------------------------

\end{document}
