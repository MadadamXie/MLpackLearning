\documentclass[11pt]{article} 

%------------------------------
%structure input 
%------------------------------
%------------------------------
%commentpackage 
%------------------------------
\usepackage{lastpage}
\usepackage{graphicx}
\setlength\parindent{0pt}
\usepackage[most]{tcolorbox}
\usepackage{booktabs}
\usepackage{etoolbox}

%------------------------------
%font settings 
%------------------------------
\usepackage[utf8]{inputenc} %required for inputting international characters,
\usepackage[T1]{fontenc} % output font encoding for international characters,
\usepackage[sfdefault,light]{roboto} %use roboto font

%------------------------------
%page setting 
%------------------------------
\usepackage{geometry}

\geometry{
  paper=a4paper,                % Change to letterpaper for US letter,        
  top=3cm,                      % Top margin,
  bottom=3cm,                   % Bottom margin,
  left=2.5cm,                   % Left margin, 
  right=2.5cm,                  % Right margin,
  headheight=14pt,              % Header height,
  footskip=1.4cm,               % Space from the bottom margin to baseline of the footer,
  headsep=1.2cm,                % Space from the top margin to the baseline of the header,
  %showframe,                   % Ucomment to show how the type block is set on the page,
}

%------------------------------
%header and footer
%------------------------------
%packages and parameters;
 \usepackage{fancyhdr}
 \pagestyle{fancy}

 % left header; 
 % output the instructor in brackets if one was set
 \lhead{\small\assignmentClass\ifdef{\assignmentClassInstructor}{\ (\assignmentClassInstructor):}{}\ \assignmentTitle} 
 \chead{}

 % right header; 
 % output the author name if one was set, otherwise the due date if that was set
 \rhead{\small\ifdef{\assignmentAuthorName}{\assignmentAuthorName}{\ifdef{\assignmentDueDate}{Due\ \assignmentDueDate}{}}} 

 % footer settings
 \lfoot{} % Left footer
 \cfoot{\small Page\ \thepage\ of\ \pageref{LastPage}} % Centre footer
 \rfoot{} % Right footer
 \renewcommand\headrulewidth{0.5pt} % Thickness of the header rule

%------------------------------
%modify section styles 
%------------------------------
\usepackage{titlesec} % required for modifying sections

%-----------------
%section settings
\titleformat
{\section}
[block]
{\Huge\bfseries}
{\assignmentQuestionName \thesection:}
{6pt}
{}

\titlespacing{\section}{0pt}{1\baselineskip}{2\baselineskip} % spacing around section titles, the order is: left, before and after

%-----------------
% subsection

\titleformat
{\subsection} % section type being modified
[block] % shape type, can be: hang, block, display, runin, leftmargin, rightmargin, drop, wrap, frame
{\huge\bfseries} % format of the whole section
{\thesubsection} % format of the section label
{4pt} % space between the title and label
{} % code before the label

\titlespacing{\subsection}{0pt}{0.5\baselineskip}{0.5\baselineskip} % spacing around section titles, the order is: left, before and after

%-----------------
% subsubsection

\titleformat
{\subsubsection} % section type being modified
[block] % shape type, can be: hang, block, display, runin, leftmargin, rightmargin, drop, wrap, frame
{\LARGE\bfseries} % format of the whole section
{\thesubsubsection} % format of the section label
{3pt} % space between the title and label
{} % code before the label

\titlespacing{\subsubsection}{0pt}{0.5\baselineskip}{0.5\baselineskip} % spacing around section titles, the order is: left, before and after



%------------------------------
%Title  
%------------------------------
\author{\textbf{\textit{\huge \assignmentAuthorName}}}
\date{}
\title{
  \thispagestyle{empty}
  \vspace{0.1\textheight}
  \textbf{\Huge\assignmentClass:\ \assignmentTitle}\\[-3pt]
  \ifdef{\assignmentDueDate}{{\large Due\ on\ \assignmentDueDate}\\}{}
  \ifdef{\assignmentClassInstructor}{{\large \textit{\assignmentClassInstructor}}}{}
  \vspace{0.32\textheight}
}

%------------------------------
%lstlisting setting 
%------------------------------
%\usepackage[T1]{fontenc}
%\usepackage{textcomp}
\usepackage{listings}
% 在LaTex中添加代码
\usepackage{color}
 
\definecolor{codegreen}{rgb}{0,0.6,0}
\definecolor{codegray}{rgb}{0.8,0.8,0.8}
\definecolor{codepurple}{rgb}{0.58,0,0.82}
\definecolor{backcolour}{rgb}{0.95,0.95,0.92}

\lstset{
  backgroundcolor=\color{backcolour},
  commentstyle=\color{codegreen},
  %keywordstyle=\color{magenta},
  stringstyle=\color{codepurple},
  basicstyle=\sffamily\footnotesize,
  breakatwhitespace=false,
  breaklines=true,
  captionpos=b,
  keepspaces=true,
  frame=single,
  numbers=left,
  numbersep=5pt,
  numberstyle=\tiny,
  showspaces=false,
  showstringspaces=false,
  showtabs=false,
  tabsize=2
}

%------------------------------
%newcommand and environment
%------------------------------
\newcommand{\answer}[1]{
  \begin{tcolorbox}[breakable,enhanced,colback=codegray]
    #1
  \end{tcolorbox}
}

\newenvironment{question}{
   \vspace{0.5\baselineskip} %whitespace before the question
   \section{} %blank section title (e.g. just Question 2)
   % set the left footer to state the question continues on the next page, this is reset to nothing 
   \lfoot{\small\itshape\assignmentQuestionName~\thesection~continued on next page\ldots} 
 }{
   \lfoot{} % reset the left footer to nothing if the current question does not continue on the next page
}

%------------------------------
%Tikz & PGF Setting
%------------------------------
\usepackage{tikz,pgfplots}
\usepackage{subcaption}

\pgfplotsset{
  standard/.style={
   axis x line=middle,
   axis y line=middle,
   enlarge x limits=0.15,
   enlarge y limits=0.15,
   %every axis x label/.style={at={(current axis.right of origin)},anchor=north west},
   %every axis y label/.style={at={(current axis.above origin)},anchor=east},
   every axis plot post/.style={mark options={fill=black}}
  }
}


%------------------------------
%Equation Order Setting to prefix the equation order.
%------------------------------
\makeatletter
\@addtoreset{equation}{section}
\makeatother
\renewcommand{\theequation}{\arabic{section}.\arabic{equation}} 

%------------------------------
%Chinese Environment 
%------------------------------
\usepackage{CJK}
\newcommand{\chinesebox}[1]{
  \begin{tcolorbox}[breakable,enhanced,colback=codegray]
   \begin{CJK*}{UTF8}{gbsn}
      #1
   \end{CJK*} 
  \end{tcolorbox}
}

\newcommand{\chinese}[1]{
 \begin{CJK*}{UTF8}{gbsn}
   #1
 \end{CJK*}
}


%------------------------------
%Multirows to Done things: 
%------------------------------
\usepackage{multirow}
\renewcommand{\multirowsetup}{\centering}


%------------------------------
%Ams math definition
%------------------------------
\usepackage{amsmath}


%------------------------------
%Frame definitions
%------------------------------
\usepackage{tcolorbox} 


%------------------------------
%Table setting
%------------------------------
\usepackage{array}
\renewcommand\arraystretch{1}





%------------------------------
%Input Parameters 
%------------------------------
% Required
\newcommand{\assignmentQuestionName}{Part}
\newcommand{\assignmentClass}{CMake} % Course/class
\newcommand{\assignmentTitle}{CM\ \#1} % Assignment title or name
\newcommand{\assignmentAuthorName}{XieYuhan} % Student name

% Optional (comment lines to remove)
\newcommand{\assignmentClassInstructor}{XieYuhan 17:34 } % Intructor name/time/description
\newcommand{\assignmentDueDate}{2023-12-29 } % Due date

\begin{document}


\maketitle

\Large

%------------------------------
%Basics sec1
%------------------------------
\clearpage
\section{Basics}%
\label{sec:basics}

From this doc we learn how to use cmake to manage our project.

\vspace{1cm}

Following are the test environment and files:

\begin{center}
  \begin{tabular}{|l|l|}
    \hline
      \textbf{File} & \textbf{}\\
    \hline
      \textbf{cmake/}hello.cpp &  \\
    \hline
      \textbf{cmake/}CMakeLists.txt &  \\
    \hline
      \textbf{cmake/}\textbf{hello}/hello1.cpp &   \\
    \hline
      \textbf{cmake/}\textbf{hello}/CMakeLists.txt &  \\
    \hline
  \end{tabular}
\end{center}


Following are some basic codes to generate the \textbf{hello\_bin}:
\begin{lstlisting}[language=make,label=lst:1lst,caption=Frame of the \textbf{cmake/CMakeLists.txt}]
  # Indicating the minimum required cmake version
  cmake_minimum_required(VERSION 2.8)

  # Set project name
  project(hello C CXX)

  # Enter into the sub-folders 
  #   which will enter into the current_dir/hello 
  #   to get the CMakeLists.txt there;
  add_subdirectory(hello)

  # Define a executable target, relying hello.cpp
  #   which means that the hello_bin will relying on
  #   source file hello.cpp(in current direction);
  add_executable(hello_bin hello.cpp) 

  # Indicating that hello_bin relying on another 
  #   target, usually to be library target;
  add_dependencies(hello_bin hello1)
\end{lstlisting}

Next file is called by \textbf{cmake/CMakeLists.txt}, 
\begin{lstlisting}[language=make,label=lst:2lst,caption=Frame of the \textbf{cmake/hello/CMakeLists.txt}]
# Use a variable to include all the related files;
set(SOURCES 
  hello1.cpp
)

# Use the following code to map the relative path of file to 
#   absolute path, storing in DIR_SRCS;
set(DIR_SRCS)
foreach(file \${SOURCES})
  set(DIR_SRCS ${DIR_SRCS} ${CMAKE_CURRENT_SOURCE_DIR}/${file})
endforeach()

# Modify the parent scope variable;
set(HELLO_SRC ${HELLO_SRC} ${DIR_SRCS} PARENT_SCOPE)

# 
add_library(hello1 hello1.cpp)

target_link_libraries(hello1
  ${ARMADILLO_LIBRARIES}
  ${Boost_LIBRARIES}
  ${LIBXML2_LIBRARIES}
)

set_target_properties(hello1
  PROPERTIES
  VERSION 1.0
  SOVERSION 1
)
\end{lstlisting}

%------------------------------
%end of Basics sec1
%------------------------------


%------------------------------
%Parameter Passing sec2
%------------------------------
\clearpage
\section{Parameter}%
\label{sec:parameter}

%------------------------------
%Parameter Passing sub2.1
\subsection{Parameter Passing}%
\label{sub:parameter_passing}

Usually, we use following frame to assign a global variable:

\textbf{Firstly}, define a variable in the parent scope; 

\textbf{Then}, call the CMakeLists.txt in the path of child directory;

\textbf{Finally}, use \textbf{set(<VAL>, PARENT\_SCOPE)} to modify the variables in parent scope.
\begin{lstlisting}[language=make,label=lst:3lst,caption=Parameter passing]
  # In parent cmake/CMakeLists.txt
  set(PROJECT_SRC,"")
  # pos1
  ...   
  add_subdirectory(hello)
  # pos2
\end{lstlisting}

\begin{lstlisting}[language=make,label=lst:4lst,caption=Parameter passing]
  # In child cmake/hello/CMakeLists.txt
  set(PROJECT_SRC, "hello1.cpp", PARENT_SCOPE)
  # pos3
\end{lstlisting}

Following are corresponding value of variable \textbf{PROJECT\_SRC}:

\begin{center}
  \begin{tabular}{|c|c|}
    \hline
     \textbf{Pos} & \textbf{Val}  \\
    \hline
     1 & ""  \\
    \hline
     2 & "hello1.cpp"  \\
    \hline
     3 & ""  \\
    \hline
  \end{tabular}
\end{center}

%-----end of Parameter Passing sub2.1

%------------------------------
%Parameter expand sub2.2
\indent
\subsection{Parameter Extending}%
\label{sub:parameter_expand}

Use \textbf{set(VAR \$\{VAR\} VAL)} to extend a variable,  following code define a empty variable \textbf{DIR\_SRCS}, then it use a \textbf{foreach()} loop to extend this variable:

\begin{lstlisting}[language=make,label=lst:4lst,caption=Extend a Parameter]
  # define a file list
  set(SOURCES
    hello1.cpp
    hello1.hpp
  )

  # define an empty variable
  set(DIR_SRCS)

  # use file list to extend the empty parameter
  foreach(file ${SOURCES})
    set(DIR_SRCS ${DIR_SRCS} ${CMAKE_CURRENT_SOURCE_DIR}/${file})
  endforeach()
\end{lstlisting}

%-----end of Parameter expand sub2.2


%------------------------------
%end of Parameter Passing sec2
%------------------------------


%------------------------------
%Target sec3
%------------------------------
\clearpage
\section{Target}%
\label{sec:target}

%------------------------------
%Target declaration sub3.1
\subsection{Target declaration}%
\label{sub:target_declaration}

Set a target generating binary ,library or set a custom target without output file:

\begin{center}
  \begin{tabular}{|c|c|}
    \hline
     \textbf{Function} & \textbf{File} \\
    \hline
     add\_executable & generate binary file \\
    \hline
     add\_library & generate lib file  \\
    \hline
     add\_custom\_target & no output file (Makefile .PONY) \\
    \hline
  \end{tabular}
\end{center}

Set a target outputing binary file:

\begin{lstlisting}[language=make,label=lst:6lst,caption=Target declaration]
  # Binary output
  add_executable(cf  # Target name;
    cf_main.cpp      #    source files dependences;
  )

  # Libraries
  add_library(mlpack # Target name;
    ${ML_SRCS}       #    source files dependences;
  )

  # Custom target
  add_custom_target(mlpack_header) # Target$name,this target won't generate a file;
\end{lstlisting}

%-----end of Target declaration sub3.1

%------------------------------
%Target\&Command sub3.2
\indent
\subsection{Target\&Command}%
\label{sub:target&command}

Bind command to target, using \textbf{add\_custom\_command}, any target is supported(no matter the target will generate an output file or not):
\begin{lstlisting}[language=make,label=lst:7lst,caption=Target declaration]
# Declare a custom target;
add_custom_target(mlpack_headers)
# Bind a command to the target;
add_custom_command(TARGET mlpack_headers POST_BUILD #Target name and build event;
  COMMENT "Moving header files to include/mlpack/"  # Comment;
  # Using cmake full path and option execute to run command;
  COMMAND ${CMAKE_COMMAND} ARGS -E 
    make_directory ${CMAKE_BINARY_DIR}/include/mlpack/)  

\end{lstlisting}
%-----end of Target\&Command sub3.2

%------------------------------
%Target\&Relations sub3.3
\subsection{Target\&Relations}%
\label{sub:target&relations}

There several kinds of relations between targets: 

1. Target  --- Source files;

2. Target --- Target;

2. Target --- Libraries;




%-----end of Target\&Relations sub3.3

%------------------------------
%end of Target sec3
%------------------------------


\end{document}
